\RequirePackage{snapshot}
%%%%%%%%%%%%%%%%%%%%%%%%%%%%%%%%%%%%%%%%%%%%%%%%%%
% Basic setup. Most papers should leave these options alone.
\documentclass[fleqn,usenatbib]{mnras}

% MNRAS is set in Times font. If you don't have this installed (most LaTeX
% installations will be fine) or prefer the old Computer Modern fonts, comment
% out the following line
\usepackage{newtxtext,newtxmath}
% Depending on your LaTeX fonts installation, you might get better results with one of these:
%\usepackage{mathptmx}
%\usepackage{txfonts}

% Use vector fonts, so it zooms properly in on-screen viewing software
% Don't change these lines unless you know what you are doing
\usepackage[T1]{fontenc}

% Allow "Thomas van Noord" and "Simon de Laguarde" and alike to be sorted by "N" and "L" etc. in the bibliography.
% Write the name in the bibliography as "\VAN{Noord}{Van}{van} Noord, Thomas"
\DeclareRobustCommand{\VAN}[3]{#2}
\let\VANthebibliography\thebibliography
\def\thebibliography{\DeclareRobustCommand{\VAN}[3]{##3}\VANthebibliography}


%%%%% AUTHORS - PLACE YOUR OWN PACKAGES HERE %%%%%

% Only include extra packages if you really need them. Common packages are:
\usepackage{graphicx}	% Including figure files
\usepackage{amsmath}	% Advanced maths commands
\usepackage{fontawesome} % GitHub icon
\usepackage{physunits}  % units
\usepackage{csvsimple}

%%%%%%%%%%%%%%%%%%%%%%%%%%%%%%%%%%%%%%%%%%%%%%%%%%

%%%%% AUTHORS - PLACE YOUR OWN COMMANDS HERE %%%%%

\newcommand{\Msun}{M$_{\odot}$} % Msun (solar mass)
\newcommand{\microgauss}{$\mu$G} % microgauss
\newcommand{\vc}[1]{{\mathbf{#1}}}
\newcommand{\hatn}{\hat{\mathbf{n}}}
\newcommand{\quokka}{\textsc{Quokka}}

% Workaround to fix the bug that prevents autoref from handling appendices properly
\newcommand{\aref}[1]{\hyperref[#1]{Appendix~\ref{#1}}}

% Autoref section names
\def\sectionautorefname{Section}
\def\subsectionautorefname{Section}
\def\subsubsectionautorefname{Section}
\def\paragraphautorefname{Section}
\def\figureautorefname{Figure}
\def\tableautorefname{Table}
\def\equationautorefname{equation}
\def\appendixautorefname{Appendix}

% Journal macros
\newcommand{\jcompphys}{J.~Comp.~Phys.}
\newcommand{\joss}{J.~Open Source Software}

% Citation alias
\defcitealias{Colella_1984}{CW84}

% MRK's editing macros
\usepackage{color}
\usepackage[normalem]{ulem}
\definecolor{darkgreen}{rgb}{0.13, 0.55, 0.13}
\newcommand{\red}[1]{{\textcolor{red}{#1}}}
\newcommand{\cyan}[1]{{\textcolor{cyan}{#1}}}
\newcommand{\mrkcut}[1]{{\red{\sout{#1}}}}
\newcommand{\mrkadd}[1]{{\cyan{#1}}}
\newcommand{\mrknote}[1]{{\textcolor{darkgreen}{[MRK: #1]}}}

%%%%%%%%%%%%%%%%%%%%%%%%%%%%%%%%%%%%%%%%%%%%%%%%%%

%%%%%%%%%%%%%%%%%%% TITLE PAGE %%%%%%%%%%%%%%%%%%%

% Title of the paper, and the short title which is used in the headers.
% Keep the title short and informative.
\title[Two-moment AMR radiation hydrodynamics on GPUs]{\textsc{Quokka}: A code for two-moment AMR radiation hydrodynamics on GPUs}

% The list of authors, and the short list which is used in the headers.
% If you need two or more lines of authors, add an extra line using \newauthor
\author[B. D. Wibking et al.]{
    Benjamin D. Wibking$^{1}$\thanks{E-mail: ben.wibking@anu.edu.au (BDW)}
    and Mark R. Krumholz$^{1}$
\\
% List of institutions
$^{1}$Research School of Astronomy \& Astrophysics, Mount Stromlo Observatory, Cotter Road, Weston Creek, ACT 2611 Australia\\
}

% These dates will be filled out by the publisher
\date{Accepted XXX. Received YYY; in original form ZZZ}

% Enter the current year, for the copyright statements etc.
\pubyear{2021}

% Don't change these lines
\begin{document}
\label{firstpage}
\pagerange{\pageref{firstpage}--\pageref{lastpage}}
\maketitle

% Abstract of the paper
\begin{abstract}
    We present \quokka, a new subcycling-in-time, block-structured adaptive mesh refinement (AMR) radiation hydrodynamics code for graphics processing units (GPUs). \quokka~solves the equations of hydrodynamics with the piecewise parabolic method (PPM) in a method-of-lines formulation, and handles radiative transfer via the variable Eddington tensor (VET) radiation moment equations with a local closure. In order to optimise GPU performance, we combine explicit-in-time evolution of the radiation moment equations with the reduced speed-of-light approximation. We show results for a wide range of test problems for hydrodynamics, radiation, and coupled radiation-hydrodynamics. On uniform grids in 3D, we achieve a peak of $69$ million hydrodynamic updates per second per GPU, and almost 10 million radiation-hydrodynamic updates per second per GPU. For radiation-hydrodynamics problems on uniform grids in 3D, our code also scales from 4 GPUs to 256 GPUs with an efficiency of $>93$ per cent. The code is publicly released under an open-source license on \faGithub\href{https://github.com/BenWibking/quokka-code}{GitHub}.
\end{abstract}

% Select between one and six entries from the list of approved keywords.
% Don't make up new ones.
\begin{keywords}
radiation hydrodynamics -- numerical methods
\end{keywords}

%%%%%%%%%%%%%%%%%%%%%%%%%%%%%%%%%%%%%%%%%%%%%%%%%%

%%%%%%%%%%%%%%%%% BODY OF PAPER %%%%%%%%%%%%%%%%%%

\section{Introduction}

%%% MRK version %%%

\subsection{Methods for radiation-hydrodynamics}

In many astrophysical systems, the radiation field carries a substantial portion of the total momentum and energy budget, and therefore must be included in any numerical simulation. However, simulating such systems, particularly at high resolution, presents a fundamental challenge in both physics and numerics. Part of this challenge is dimensionality: in full generality the radiation field is governed by the time-dependent equation of radiative transfer,
\begin{equation}
\frac{1}{c} \frac{\partial}{\partial t} I_\nu + \hatn\cdot\nabla I_\nu = \eta_\nu - \kappa^{\rm (tot)}_\nu \rho I_\nu + \frac{1}{4\pi} \int \kappa^{\rm (sca)}_\nu \rho I_\nu \, d\Omega,
\label{eq:transfer}
\end{equation}
where $I_\nu$ is the radiation intensity at frequency $\nu$ travelling in the direction specified by the unit vector $\hatn$, $\eta_\nu$ is the matter emissivity, $\rho$ is the matter density, and $\kappa_\nu^{\rm (tot)}$ and $\kappa_\nu^{\rm (sca)}$ are the total and scatting specific opacities, respectively. This is a time-dependent integro-differential equation with six dimensions: three positions, two angles (parameterised by $\hatn$), and the frequency. Full numerical solution of a problem of this dimensionality, at least if it must be done millions of times to run in tandem with a hydrodynamic (HD) or magnetohydrodynamic (MHD) simulation, remains out of reach for most applications. 

Within the astrophysics community, there are two general approaches to this problem. One is characteristic methods, which solve \autoref{eq:transfer} (or its time-independent form), but only along rays pointing back to particular sources \citep[e.g.,][]{Abel02a, Rijkhorst06a, Krumholz07f} or rays randomly assigned by Monte Carlo \citep[e.g.,][]{Fleck71a, Tsang15a}. A second approach, which we will pursue here, is moment methods \citep[e.g.,][]{Mihalas_1984,Castor_2004}, whereby one takes moments of the transfer equation, thereby eliminating the angular dimensions of the problem.\footnote{Fully eliminating the angular dependence requires some care, since in general for a moving medium the emissivity and opacity depend on direction, even if the medium itself interacts with light isotropically in its own rest frame. While one might guess velocity-dependent beaming effects are unimportant in non-relativistic problems, it turns out that one cannot formulate a consistent, energy-conserving theory of non-relativistic radiation-hydrodynamics without including them, at least to leading order \citep{Mihalas82a, Lowrie99a, Krumholz_2007}. Systems of moment equations where the radiation moments are written in the lab frame but the emissivity and absorption are written in the comoving frame, where they can be assumed to be isotropic, are known as mixed-frame formulations. This is the most common approach in modern radiation-hydrodynamics codes.} This results in a series of moment equations that one can solve in lieu of solving the equation of radiative transfer directly, but at the price of introducing the need for a closure relation to approximate the higher moments that appear in the equations being solved. Some authors also combine moment and characteristic methods \citep[e.g.,][]{Rosen17a}.\footnote{While characteristic and moment methods are the only ones widely used in astrophysics, in terrestrial applications (for example neutron transport calculations for nuclear reactor design) there are two other widely-used approaches to handle the angular dependence of the transfer equation. One is to discretise the unit sphere using weighted quadratures (the discrete ordinates, or $S_n$, method; e.g., \citealt{Lathrop_1964,Adams97a}). The other is to expand the angular integration in spherical harmonics (the $P_n$ method; e.g., \citealt{Modest89a}).}

One of the simplest closures is the flux-limited diffusion (FLD) approximation \citep{Alme_1973}, which retains only the first moment equation, for the radiation energy density, and closes the system by assuming both that the time derivative of the radiation flux is zero and that the Eddington tensor, defined as the ratio of the radiation pressure tensor to the radiation energy density, has a fixed value. A more accurate approximation is to evolve both the radiation energy density and the radiation flux, while still invoking a closure relation for the Eddington tensor; this is called a two-moment approach, since one solves for the first two moments of the radiation field. When the Eddington tensor is computed via a formal solution of the angle-dependent radiative transfer equation, we obtain the quasidiffusion or variable Eddington tensor (VET) method \citep{Goldin_1964}. When retaining a local closure for the radiation pressure tensor in terms of the radiation energy density $E$ and the flux $F$, we obtain a local VET method, commonly referred to as the M1 (`moment-one`) method \citep{Minerbo_1978,Levermore_1984,Dubroca_1999,Gonzalez_2007}. There are a number of moment-based astrophysical radiation-hydrodynamic codes, implementing a wide variety of closures, in wide use, including \textsc{Zeus} \citep{Turner_2001}, \textsc{FLASH} \citep{Fryxell_2000}, \textsc{Orion} \citep{Krumholz_2007, Shestakov08a, Li21a}, \textsc{Ramses} \citep{Commercon11a, Rosdahl_2013}, \textsc{Athena} \citep{Davis_2012, Jiang12a}, \textsc{Enzo} \citep{Reynolds_2009,Bryan_2014}, \textsc{Castro} \citep{Zhang_2011,Zhang_2013,Almgren_2020}, and \textsc{Fornax} \citep{Skinner_2019}, to give a partial list.

While the use of moment methods removes the dimensionality problem, it leaves a second problem, which is the strong mismatch in signal speeds between radiation and sound (or MHD) waves, which in a non-relativistic system travel at far less than the speed of light. This mismatch renders simple explicit methods, as are commonly used for HD and MHD, impractically slow for radiative transfer, due to the tiny time steps that would be imposed by the Courant-Friedrichs-Lewy (CFL) condition. For this reason, numerical methods for coupled radiation-hydrodynamics (RHD) or radiation-magnetohydrodynamics (RMHD) either use an implicit method for the radiation part of the problem (e.g., \textsc{Zeus}, \textsc{Orion}, some versions of \textsc{Ramses} and \textsc{Athena}) or adopt the reduced speed of light approximation (\citealt{Gnedin_2001, Skinner_2013}; \textsc{Fornax}, other versions of \textsc{Ramses} and \textsc{Athena}). The reduced speed of light approximation (RSLA) consists of replacing the speed of light $c$ that defines the signal speed in the radiation moment equations with a lower speed $\hat{c}$, while keeping the terms that describe the rate of momentum and energy exchange between gas and radiation unchanged. The lower speed $\hat{c}$, while still substantially larger than the HD or MHD signal speeds, is close enough to those speeds to allow radiation time steps large enough to render explicit methods computationally feasible.

\subsection{Why a new radiation-hydrodynamics code?}

In this paper we describe \quokka\footnote{Quadrilateral, Umbra-producing, Orthogonal, Kangaroo-conserving Kode for Astrophysics!}, a new code for RHD. In terms of the taxonomy introduced above, \quokka~is a two-moment code that uses the RSLA to allow an explicit treatment of radiation transport. \quokka~is also an adaptive mesh refinement (AMR) code, so it begins with a base grid at uniform spatial resolution, but then dynamically adds higher-resolution grids as needed to achieve user-specified accuracy goals \citep{Berger:1984, Berger:1989}. However, these features do not make \quokka~unique: \textsc{Orion} and \textsc{Ramses} (among others) offer moment-based AMR RMHD, while \textsc{Fornax} uses RSLA on a dendritic (though not adaptive) grid.

%%% BDW: Castro's FLD runs on GPU, so need to cite and clarify we are the first AMR two-moment RHD on GPU
The unique feature of \quokka~is that it has been designed from the ground up to run efficiently on graphics processing units (GPUs). This design goal motivated our choice of both algorithms and low-level implementation details. While \quokka~is not the first GPU hydrodynamics code in astrophysics (others include \textsc{Gamer}, \citealt{Schive10a, Schive18a}, \textsc{Cholla}, \citealt{Schneider15a}, \textsc{Castro}, \citealt{Almgren_2020}, and \textsc{ARK-RT}, \citealt{Bloch_2021}), nor even the first AMR GPU code, it is the first to feature two-moment AMR RHD on GPUs.

Bringing RHD to GPUs creates some unique challenges. Contemporary compute nodes are often limited by data bandwidth, both in terms of moving data between main memory and the CPU or GPU, and in terms of moving data between CPUs or GPUs. For this reason, implicit methods generally have poor scalability, due to the need for global communications during an implicit solve (see, e.g., Appendix E of \citealt{Skinner_2019}). This imbalance between computation and communication is magnified on GPUs. Likewise, robust implicit methods require iterative sparse matrix solvers, which achieve lower peak efficiency on GPUs compared to CPUs due to their heavy use of indirect addressing and highly branching control flow. These considerations motivate our choice of an explicit RSLA method. They also motivate our choice of time integration strategy, which as we detail below has been designed to maximize computation (and therefore minimize the relative amount of communication) on each hydrodynamic timestep. We show that, with this strategy, we are able to achieve update computation rates of approximately 70 million zone updates per second per GPU for pure HD, and almost 10 million for RHD. We also achieve $>93\%$ parallel efficiency out to 256 GPUs. This combination of performance and scaling makes \quokka~substantially faster than any other public RHD code.

The remainder of this paper is organized as follows. In \autoref{section:methods} we introduce the set of equations that \quokka~solves, and detail our numerical methods for solving them. In \autoref{section:tests}, we present a wide range of tests that demonstrate the accuracy and capabilities of the code. \autoref{section:performance} covers our tests of code performance and scalability. In \autoref{section:discussion}, we discuss the range of applicability of our methods, and our plans for application and future code expansions. Finally, the code itself, including all test problems, is freely available on \faGithub\href{https://github.com/BenWibking/quokka-code}{GitHub} under an open-source license. 

\section{Methods}
\label{section:methods}
\subsection{Equations}
We solve the equations of radiation hydrodynamics \citep{Pomraning_1973,Mihalas_1984,Castor_2004} for an inviscid, nonrelativistic fluid in local thermodynamic equilibrium in the mixed-frame formulation, where the radiation variables are defined in an inertial frame (i.e., Eulerian simulation coordinates) and the radiation-matter interaction terms are written in the frame comoving with the fluid, with the transformations between the frames accounted for via the addition of radiation-matter exchange terms that depend explicitly on the ratio of fluid velocity to the speed of light, $\beta=v/c$. In this first version of \quokka~we omit scattering, so that matter-radiation interaction is purely bey emission and absorption. We write the equations as follows:
\begin{align}
    \frac{\partial \rho}{\partial t} + \nabla \cdot (\rho \vc{v}) = 0 \, , \\
    \frac{\partial (\rho \vc{v})}{\partial t} + \nabla \cdot (\rho \vc{v} \vc{v} + \mathsf{P}) = \vc{G} \, , \\
    \frac{\partial E}{\partial t} + \nabla \cdot \left[(E + \mathsf{P})\vc{v}\right] = c G^0 \, , \\
    \frac{\partial E_r}{\partial t} + \nabla \cdot {\vc{F}_r} = -c G^0 \, , \\\
    \frac{1}{c^2}\frac{\partial \vc{F}_r}{\partial t} + \nabla \cdot \mathsf{P}_r = -\vc{G} \, ,
\end{align}
where $\rho$ is the gas density, $\vc{v}$ is the gas velocity, $E$ is the total energy density of the gas, $\mathsf{P} = \delta_{ij} P$ is the gas pressure tensor, $E_r$ is the radiation energy density, $F_r$ is the radiation flux, $\mathsf{P}_r$ is the radiation pressure tensor, $\nabla \cdot \rho \vc{v} \vc{v}$ denotes the sum $(\rho v_i v^j)_{,j}\,$, and $G^i$ is the radiation four-force, with $G^0$ the time-like component and $\vc{G}$ consisting of the space-like components. In the mixed-frame formulation, the radiation four-force to order $\beta$ is
\begin{align}
-c G^0 = \rho (\kappa_P 4 \pi B - \kappa_E c E_r) + \rho \kappa_F \left( \frac{\vc{v}}{c} \cdot \vc{F}_r \right) \, , \\
-\vc{G} = -\rho \kappa_F \frac{\vc{F}_r}{c} + \rho \kappa_P \left(\frac{4 \pi B}{c}\right) \frac{\vc{v}}{c} + \rho \kappa_F \frac{\vc{v}\mathsf{P}_r}{c} \, ,
\end{align}
where
$\kappa_F$, $\kappa_E$, and $\kappa_P$ are the flux-mean, energy-mean, and Planck-mean specific opacities evaluated in the comoving frame,
$B$ is the Planck function evaluated at the gas temperature,
and $\vc{v} \mathsf{P}_r$ is the tensor contraction $v_j \mathsf{P}_r^{ij}$ \citep{Mihalas_1984}. The latter two terms in the expression for $\vc{G}$ correspond to the relativistic work term of \cite{Krumholz_2007} and are only important in the regime $\beta \tau \gtrsim 1$ (where $\tau$ is a characteristic optical depth), 
to which we cannot apply the RSLA (as discussed below),
so we neglect them. However, the term of order $\beta$ in the expression for $cG^0$ corresponds to the work done by the radiation force on the gas and can be the dominant term for problems of interest. 

To apply the RSLA to these equations, we first rewrite the radiation moment equations so that they have a factor of exactly $1/c$ next to each of the time derivatives:
\begin{align}
    \frac{1}{c} \frac{\partial E_r}{\partial t} + \nabla \cdot \left( \frac{\vc{F}_r}{c} \right) = -G^0 \, , \\\
    \frac{1}{c} \frac{\partial}{\partial t} \left( \frac{\vc{F}_r}{c} \right) + \nabla \cdot \mathsf{P}_r = -\vc{G} \, ,
\end{align}
then we replace this $1/c$ factor with a factor of $1/\hat c$, where $\hat c$ is the reduced speed of light, and multiply through by factors of $\hat c$ to obtain the conservation law form of the reduced speed of light radiation moment equations (e.g., \citealt{Skinner_2013}):
\begin{align}
    \frac{\partial E_r}{\partial t} + \nabla \cdot \left( \frac{\hat c}{c} \vc{F}_r \right) = -\hat c G^0 \, , \\\
    \frac{\partial \vc{F_r}}{\partial t} + \nabla \cdot (c \hat c \, \mathsf{P}_r) = -c \hat c \, \vc{G} \, .
\end{align}
The maximum wave speed of this system of equations is bounded by $\hat c$ (as long as the flux satisfies causality, i.e. $F_r \leq cE_r$). As emphasised by \cite{Skinner_2013}, all other factors of $c$ remain unchanged, and, since the factors of $c$ are unchanged on the right-hand side of the hydrodynamic equations, the reduced speed of light radiation hydrodynamic system does not conserve total energy or momentum for $\hat{c} \neq c$. 
When the left-hand side flux divergence terms are negligible, this nonconservation implies that the equilibrium temperature of the reduced speed of light system is slightly modified with respect to the correct equilibrium temperature, implying that we cannot apply the RSLA to problems in the equilibrium diffusion limit in general (see section \ref{section:equilibrium}).

Writing out the right-hand side terms explicitly, we obtain
\begin{align}
    \label{eq:hydro_continuity}
    \frac{\partial \rho}{\partial t} + \nabla \cdot (\rho \vc{v}) = 0 \, , \\
    \label{eq:hydro_momentum}
    \frac{\partial (\rho \vc{v})}{\partial t} + \nabla \cdot (\rho \vc{v} \vc{v} + \mathsf{P}) = \rho \kappa_F {\vc{F}_r / c} \, , \\
    \label{eq:hydro_energy}
    \frac{\partial E}{\partial t} + \nabla \cdot \left[(E + \mathsf{P})\vc{v}\right] = -c \rho (\kappa_P a_r T^4 - \kappa_E E_r) - \rho \kappa_F \left( \frac{\vc{v}}{c} \cdot \vc{F}_r \right) \, , \\
    \label{eq:rad_energy}
    \frac{\partial E_r}{\partial t} + \nabla \cdot \left( \frac{\hat c}{c} \vc{F}_r \right) = \hat c \rho \left(\kappa_P a_r T^4 - \kappa_E E_r \right) + \rho \kappa_F \left( \frac{\hat c}{c} \frac{\vc{v}}{c} \cdot \vc{F}_r \right) \, , \\\
    \label{eq:rad_flux}
    \frac{\partial \vc{F}_r}{\partial t} + \nabla \cdot (c \hat c \, \mathsf{P}_r) = -\hat c \rho \kappa_F \vc{F}_r \, .
\end{align}
These equations make no approximations about the frequency dependence of the radiation field. However, for computational tractability, in what follows we will approximate $\kappa_F$ with the Rosseland mean opacity $\kappa_R$, which yields the correct radiation force in the diffusion limit, and approximate $\kappa_E$ with the Planck mean opacity $\kappa_P$, which yields the correct energy absorption and emission in the optically-thin limit for fluids at rest \citep{Mihalas_1984}. However, we emphasise that the choice to set $\kappa_F \approx \kappa_R$ and $\kappa_E \approx \kappa_P$ is an additional approximation, and that others might be preferable depending on the physical system being simulated.  In future work, we plan to address the limitations of these approximate grey opacities via an extension of our method to the multigroup solution of the radiation moment equations.  Our present set of equations is sufficient for grey nonrelativistic radiation hydrodynamics in the semi-transparent regime, where we can neglect the `relativistic work term' that is important only in the dynamic diffusion ($\beta \tau \gtrsim 1$) regime, as described earlier.

\subsection{Solution method}

% MRK I thought we needed an overview paragraph to explain the structure of the solve overall, so I added a parent section here, and each of the previous sections to sub-sections.
We solve the system formed by \autoref{eq:hydro_continuity}--\autoref{eq:rad_flux} using an operator split approach, whereby we first advance the hydrodynamic transport subsystem (\autoref{sssec:hydro}), then the radiation transport subsystem (\autoref{sssec:radiation}), and finally update the local coupling terms (\autoref{sssec:coupling}). The first subsystem uses a single explicit update step, the second a set of subcycled explicit updates, and the third a purely local implicit update. We describe each of these steps below.

This update cycle operates within a \citet{Berger:1984} / \citet{Berger:1989} adaptive mesh refinement (AMR) framework, whereby each spatial variable is represented by a volume average in each cell, on a rectangular, Cartesian grid. We cover the entire computational domain with a coarse grid with cell spacings $\Delta x_0$, $\Delta y_0$, $\Delta z_0$ in the $x$, $y$, and $z$ directions, the grid spacings need not be the same, but for most applications we choose them to be the same. We denote this coarse grid level 0. We then dynamically add (or remove) additional, finer grids over parts of the domain in response to user-specified refinement criteria. We denote these additional levels 1, 2, $\ldots$, with each grid on level $l$ having cells a factor of $2$ smaller than those on level $l-1$, so that the cell spacing on level $l$ is $\Delta x_0/2^l$ in the $x$ direction, and similarly for $y$ and $z$. We use only factor of 2 refinements in order to minimize numerical glitches arising from the discontinuous change in resolution, which can arise especially in problems where shocks cross the coarse-fine mesh interface at an oblique angle (e.g., \citealt{Fryxell_2000}). When adding finer grids we conservatively interpolate the underlying coarse data, and when removing finer grids we conservatively average down the fine data. Time steps on different AMR levels are sub-cycled, such that the time step on level $l$ is $\Delta t_l = \Delta t_0/2^l$. At the end of every two time steps on level $l > 0$, we perform a synchronization step to ensure that we maintain machine-precision conservation for all conserved quantities (\autoref{sssec:sync}). Our implementation of AMR in \quokka~is based on the \textsc{AMReX} library \citep{AMReX_JOSS, the_amrex_development_team_2021_5363443}.

\subsubsection{Hydrodynamics}
\label{sssec:hydro}
For the solution of the hydrodynamic subsystem (\autoref{eq:hydro_continuity}--\autoref{eq:hydro_energy}, omitting the matter-radiation coupling terms on the right hand sides), we adopt a method-of-lines (or semi-discrete) approach, discretizing the spatial variables while initially keeping the time variable continuous, thereby transforming the partial differential equations into a large set of ordinary differential equations that can be integrated in time using a standard ordinary differential equation (ODE) integrator \citep{Jameson_1981}. For the latter, we use the second-order strong stability preserving Runge-Kutta method (RK2-SSP; \citealt{Shu_1988}). Such an approach has been successfully employed in several recent astrophysical hydrodynamics codes \citep{Skinner_2019,Stone_2020}. Somewhat surprisingly, we find that such a method-of-lines scheme is not stable when combining higher-order spatial reconstruction with forward Euler time integration, but is stable for timesteps $\mathcal{O}(\Delta x / (c_s + |v|))$ when used with higher-order Runge-Kutta methods. This is consistent with the findings of \cite{Stone_2020}, who also find that the maximum stable timestep corresponding to the linear stability region of a given (second order or higher) Runge-Kutta integrator divided by the number of spatial dimensions. As \cite{Skinner_2019} notes, in contrast to fully-discrete unsplit hydrodynamic methods such as the corner transport upwind (CTU) method \citep{Colella_1990}, the coupling across corners of each cell is achieved via the use of a multi-stage time integrator, rather than via direct computation of fluxes from diagonal neighbors of each cell. While we are formally limited to a smaller timestep compared to the CTU method, our method may be more robust in practice, as the CTU integrator has been found to be unstable in supersonic turbulence with strong radiative cooling unless very small ($\lesssim 0.1$) CFL numbers are employed \citep{Schneider_2017}.

We reconstruct the hydrodynamic variables on each face of each cell from the cell-average variables of the neighbouring cells. We perform this reconstruction using the piecewise parabolic method (PPM) \citep[hereafter \citetalias{Colella_1984}]{Colella_1984} using the primitive hydrodynamic variables (density, velocity, and pressure). As is standard, the conversion from conserved (density, momentum, and energy) to primitive variables is carried out assuming that the volume average and cell centered states are equivalent, which is an approximation accurate to $\mathcal{O}(\Delta x^2)$. As noted by several authors, the PPM algorithm is therefore formally second-order accurate in spatial resolution. After the primitive variables have been defined, for the reconstruction step proper, we use the standard interface-centered PPM stencil:
\begin{align}
q_{j+1/2} = \frac{7}{12} (q_j + q_{j+1}) - \frac{1}{12} (q_{j+2} + q_{j-1}).
\end{align}
We follow the implementation of \cite{Stone_2020} in re-grouping the above terms symmetrically with respect to the interface ${i+{1/2}}$ so as to preserve exact symmetry in floating point arithmetic. On a mesh with uniform spacing between cells along the direction perpendicular to the reconstructed face, this stencil is fourth-order accurate.

We do not perform the slope-limiting and contact steepening steps of \citetalias{Colella_1984}. We instead enforce monotonicity of the reconstructed state by re-setting the interface state to lie between the values of the cells adjacent to the interface, following \cite{Mignone_2005}. This is followed by the extrema detection and overshoot correction step within each cell as described by \citetalias{Colella_1984}. In this step, the parabola assumed to exist across each cell is examined. If an `overshoot' (as defined by \citetalias{Colella_1984}) of the parabola is detected, we follow the original \citetalias{Colella_1984} prescription of performing linear reconstruction on the side of the cell affected by the overshoot. If an extremum is instead detected, rather than forcing the reconstruction to a constant value across the cell as done by \citetalias{Colella_1984}, we revert to performing a linear reconstruction within the affected cell, following \cite{Balsara_2017}. Either of these steps may make the interface states discontinuous, with distinct states associated with each of the two cells adjacent to an interface, which reduces the order of accuracy of the stencil. Since the overall algorithm is a combination of second order and (usually) fourth order steps, the overall order of accuracy of PPM is often referred to as `third order' (e.g., \citealt{Stone_2020}). However, we note that there exist fully fourth-order extensions of PPM \citep{Felker_2018}, but these methods do not allow local source terms to be evaluated independently for each cell while maintaining fourth-order accuracy.

We also implement reconstruction based on a piecewise-linear method (PLM) using the monotonized-central (MC) slope limiter \citep{VanLeer_1977}. We use PPM reconstruction by default, but allow PLM reconstruction via a compile-time option.

In some cases, especially in underresolved strong shocks, the previous steps do not provide sufficient dissipation to avoid oscillations. This problem was recognized by \citetalias{Colella_1984}, who proposed a shock flattening procedure in combination with a small amount of artificial viscosity. We find that this shock flattening procedure is not sufficient in multidimensional problems. Instead, we follow \cite{Miller_2002}, who generalize the \citetalias{Colella_1984} procedure for multidimensional hydrodynamics. Using this latter method, we find that no artificial viscosity is needed and we do not include any in our implementation.

Finally, in order to compute the flux of mass, momentum and energy between cells, we use the HLLC Riemann solver with the `primitive variable Riemann solver' wavespeeds and intermediate states \citep{Toro_2013}. We make the standard approximation that the face-average flux is the same as the face-centered flux, and therefore this step is also second-order accurate in spatial resolution. For each cell, the fluxes across each face are then added together to produce an unsplit spatial divergence term used by each stage of the Runge-Kutta integrator to advance the cell in time.

In multidimensional simulations, it has been long recognized that in strong grid-aligned shocks, the HLLC Riemann solver can unphysically amplify the so-called `carbuncle' instability \citep{Quirk_1994}. In astrophysical problems, this is most often encountered in strong explosions. Implementing additional dissipation in the form of artificial viscosity (e.g., \citealt{Gittings_2008}), the `H-correction' \citep{Sanders_1998}, or by adaptively switching to an HLL Riemann solver \citep{Harten_1983} for computing fluxes perpendicular to strong shocks (e.g., \citealt{Quirk_1994,Skinner_2019}) are possible solutions to this issue. In future work we plan to implement an adaptive procedure to fix the carbuncle instability via the latter method.

Future work may also include implementing an adaptive method to reduce the order of reconstruction in order to preserve density and pressure positivity in near-vacuum regions, such as the multidimensional optimal order detection (MOOD) method of \cite{Clain_2011}. An alternative solution may be to adaptively switch to an exact (iterative) Riemann solver depending on the flow conditions \citep{Toro_2013}.

\subsubsection{Radiation}
\label{sssec:radiation}

We solve the radiation transport subsystem (\autoref{eq:rad_energy}--\autoref{eq:rad_flux}, again omitting the terms on the right-hand side) in a similar method-of-lines fashion. Our approach is most similar to that of \cite{Skinner_2019}, who also evolve the radiation moment equations with a time-explicit method-of-lines approach; however, they do not use either PPM reconstruction or a reduced speed of light. Because even with the RSLA the signal speed for the radiation subsystem is substantially larger than for the hydrodynamic subsystem, we evolve the former explicitly in time with several radiation timesteps per hydrodynamic timestep. In the regime of applicability of the RSLA, this approach allows a much more computationally efficient solution to the radiation moment equations, due to the fact that explicit methods have a greater arithmetic intensity per byte of data, have simple memory access patterns and control flows (compared to implicit solvers), and do not require global communication across the computational domain in order to advance the solution in time. All these features are greatly beneficial on GPUs, where the ratio of floating-point arithmetic performance to memory bandwidth is typically greater than on CPUs.

We carry out each radiation subcycle using the same RK2-SSP integrator \citep{Shu_1988} that we use for hydrodynamics. We likewise use a finite volume representation of the radiation variables, with a PPM spatial reconstruction (or optionally, PLM) of the radiation energy density $E_r$ and reduced flux $\mathbf{f} = \mathbf{F}_r / cE_r$; the only difference in our procedures for hydrodynamics and radiation is that for radiation we do not employ a shock flattening procedure. There can exist unphysical radiation shocks when using local closures, since in general such closures make the radiation subsystem nonlinear, but there is no applicable shock flattening procedure to suppress this effect. We carry out reconstruction in terms of the reduced flux $\mathbf{f}$ rather than the absolute flux $\mathbf{F}_r$ in order to suppress unphysical fluxes $|\mathbf{F}_r| > cE$. This is effective in 1D problems, but in multidimensional problems, the magnitude of the radiation flux may still exceed $cE_r$, which is an unphysical state in which local closures cannot compute the Eddington factor at all. Reducing the order of reconstruction to first order (piecewise constant) when the interface states violate this constraint helps but does not eliminate the issue in all cases. For the purpose of computing the local closure only, we use rescale the flux such that $|\mathbf{F}_r| = cE_r$ whenever $|\mathbf{F}_r| > cE_r$.  For particularly difficult problems, especially in order to avoid unphysical instabilities in the propagation of non-grid-aligned optically-thin radiation fronts, we find that it is necessary to reconstruct the radiation variables using PLM reconstruction.

One drawback to upwind finite volume methods for radiation transport is that in naive form, they do not give the correct behavior for diffusive regions where the optical depth per cell is much greater than unity. This failure occurs because numerical diffusion dominates over physical diffusion when using upwind methods when the mean free path of photons is not resolved \citep{Lowrie_2001}. One common approach to fix this incorrect behavior is to modify the Riemann solver in the optically thick regime to reduce the upwind bias of the spatial derivative \citep{Audit_2002,Skinner_2019,Mezzacappa_2020}. However, this can lead to violations of causality (i.e., $|\mathbf{F}_r| > cE_r$) when the radiation flux is in the streaming regime \citep{Audit_2002}, which occurs especially at discontinuities in the opacity between optically-thin cells and optically-thick cells. The only apparent fix for this problem, which we adopt, is to disable the optical-depth correction in the Riemann solver for those cells where it produces a causality-violating state. We find that this condition is only activated when $f \rightarrow 1$, so it may not qualitatively affect the solution. However, we also advocate refining on the gradient in the optical depth per cell in order to resolve the boundary layers in such situations whenever it is computationally feasible. In \aref{appendix:asymptotic_correction}, we show that such an optical-depth correction substantially improves the behavior of the method in the diffusion regime, but is not sufficient to obtain the correct diffusion solution in thermal waves.

For computing the flux of radiation quantities between cells, we use an HLL Riemann solver, with wavespeeds computed assuming the Eddington factors are fixed at the beginning of the timestep \citep{Balsara_1999}. This approach allows us to substitute different closure relations for the Eddington factors without requiring a modification of the Riemann solver, unlike previous implementations that are restricted to a single local closure (e.g., \citealt{Gonzalez_2007,Skinner_2013,Skinner_2019}). In principle, we could even use Eddington tensors computed via a short characteristics formal solution of the radiative transfer equation (e.g., \citealt{Davis_2012}), but we leave exploration of a non-local VET method to future work.

Our default closure relation for the Eddington tensor is the \cite{Levermore_1984} closure, which is derived by assuming that the radiation field is isotropic in some (unknown) reference frame and then computing a Lorentz transform from this reference frame to one in which the reduced flux $\mathbf{f}$ matches the value in the cell under consideration. This procedure leads to a radiation pressure tensor (e.g., \citealt{Gonzalez_2007,Rosdahl_2013,Skinner_2013})
\begin{align}
\label{eq:M1_closure}
\mathsf{P}_r = \left( \frac{1 - \chi}{2} \mathsf{I} + \frac{3\chi - 1}{2} \mathbf{\hat n} \mathbf{\hat n} \right) E_r
\end{align}
where $\mathsf{I}$ is the identity tensor, and the Eddington factor $\chi$ and the flux direction cosine $\mathbf{\hat n}$ are
\begin{align}
\chi = \frac{3 + 4f^2}{5 + 2 \sqrt{4 - 3 f^2}} \, , \\
\mathbf{\hat n} = \frac{\mathbf{F}_r}{|\mathbf{F}_r|} \, .
\end{align}
When the radiation flux is exactly zero, we drop the direction-dependent term in \autoref{eq:M1_closure}. By considering a coordinate system where the radiation flux is aligned with a coordinate axis, we see that $\chi$ is the component of the Eddington tensor in the direction of the radiation flux.

We emphasise that this is only one possible choice of closure, and a variety of alternative local closures exist \citep[e.g.,][]{Minerbo_1978,Levermore_1981}. We refer readers to \citet{Janka_1992} and \citet{Koerner_1992} for systematic comparisons to angle-dependent transport solutions for neutrinos, and \citet{Olson_2000} for comparisons to photon solutions. Because of its prominence in the neutrino transport literature, as well as marginally favorable performance on some test problems, we also provide an implementation of the \cite{Minerbo_1978} closure in addition to the default \citet{Levermore_1984} option. However, users can also implement any local closure of their choice simply by providing an implementation of a function that maps from the reduced flux $\mathbf{f}$ to the Eddington factor $\chi$ for their preferred closure. Doing so does not come at any cost in computational performance.

\subsubsection{Matter-radiation coupling}
\label{sssec:coupling}

Following the computation of the hyperbolic part of the radiation subsystem, we use an implicit method to evaluate the source terms (those appearing on the right-hand sides of \autoref{eq:hydro_continuity}--\autoref{eq:rad_flux}) for both the radiation and hydrodynamic subsystems; this update occurs once per radiation subcycle, and thus several times per hydrodynamic step. Since there are no spatial derivatives in these terms, each cell can be updated independently.

The radiation-matter coupling update occurs in three steps. The first is to handle the energy source terms $c\rho (\kappa_P a_r T^4 - \kappa_E E_r)$ that appear in \autoref{eq:hydro_energy} and \autoref{eq:rad_energy}. In the regime of problems to which we can apply the RSLA this term is often the stiffest, and we therefore update it using the backward-Euler implicit method of \cite{Howell_2003}, specialized to the case of a single material and extended to include a reduced speed of light. Let $E_g = E - \rho v^2/2$ be the gas internal energy, and let $E_g^{(t)}$ and $E_r^{(t)}$ be the gas internal energy and radiation energy at the end of the hyperbolic update, where the superscript $(t)$ indicates quantities evaluated at this point in the update cycle. We compute the new gas internal energy $E_g^{(t+1)}$ and radiation energy $E_r^{(t+1)}$, where $(t+1)$ indicates the state after accounting for the exchange term, by solving the implicit system
\begin{eqnarray}
0 = F_G &\equiv &(E_g^{(t+1)} - E_g^{(t)}) + \left( \frac{c}{\hat c} \right) R^{(t+1)} \\
0 = F_R &\equiv & (E_{r}^{(t+1)} - E_{r}^{(t)}) - \left( R + S \right)^{(t+1)},
\end{eqnarray}
where 
\begin{equation}
R \equiv \Delta t \rho \kappa_P (4 \pi B - \hat c E_r),
\end{equation}
$\Delta t$ is the radiation substep timestep, and $S$ is an optional source term that we include to allow, for example, addition of radiation by stellar sources. The quantities $F_G$ and $F_R$ are the residual errors in the gas energy and radiation energy, respectively.

To solve this system via Newton-Raphson iteration, we require the Jacobian matrix, the elements of which are
\begin{align}
\frac{\partial F_G}{\partial E_g} &= 1 + \left( \frac{c}{\hat c} \right) \frac{\partial R}{\partial E_g} \, , \\
\frac{\partial F_G}{\partial E_r} &= -c \Delta t \rho \kappa_P \, , \\
\frac{\partial F_R}{\partial E_g} &= -\frac{\partial R}{\partial E_g} \, , \\
\frac{\partial F_R}{\partial E_r} &= 1 + \hat c \Delta t \rho \kappa_P \, ,
\end{align}
where
\begin{align}
\frac{\partial R}{\partial E_g} &= \frac{\rho \Delta t}{C_v} \left[ \kappa_P \frac{\partial B}{\partial T} + \frac{\partial \kappa_P}{\partial T} \left( 4\pi B - \hat c E_r^{t+1} \right) \right] \, ,
\end{align}
$C_v$ is the gas specific heat at constant volume. From the Jacobian, we can write the change in radiation and gas temperature for each iterative update as
\begin{align}
\label{eq:delta_er}
\Delta E_r &= -\frac{F_R + \eta F_G}{ \frac{\partial F_R}{\partial E_r} + \eta \frac{\partial F_G}{\partial E_r} } \, , \\ 
\label{eq:delta_eg}
\Delta E_g &= -\frac{F_G + \Delta E_r \frac{\partial F_G}{\partial E_r}}{ \frac{\partial F_G}{\partial E_g} } \, .
\end{align}
where $\eta \equiv - (\partial F_R/\partial E_g) ( \partial F_G/\partial E_g )^{-1}$. We repeatedly apply \autoref{eq:delta_er} and \autoref{eq:delta_eg} to the radiation and gas energies until the system converges. \citet{Howell_2003} leave unspecified the convergence criteria they use for their solver. After experimenting with several possibilities, we decide to stop the Newton-Raphson iterations when the residuals $F_R$ and $F_G$ satisfy
\begin{align}
\left| \frac{F_G}{E_{\text{tot}}} \right| &< \epsilon \, \, \text{and} \\
\left| \frac{c}{\hat c} \frac{F_R}{E_{\text{tot}}} \right| &< \epsilon \, ,
\end{align}
where
\begin{align}
E_{\text{tot}} &\equiv E_g^{(t)} + \frac{c}{\hat c} \left( E_r^{(t)} + S \right) \, .
\end{align}
When $\hat c = c$, $E_{\text{tot}}$ is the total (internal gas plus radiation) energy at the end of the timestep. By default, the relative tolerance $\epsilon$ is set to $10^{-10}$. We find that larger tolerances produce unacceptably inaccurate solutions for many problems. In especially stiff problems, it may be necessary to reduce the tolerance to the order of machine precision for double-precision floating point arithmetic ($\sim 10^{-15}$). If the solver exceeds a specified maximum number of iterations (400 by default) without converging, the code prints an error message and exits. Convergence failure usually occurs only when the intial timestep has not been sufficently reduced compared to the CFL timestep at the start of a simulation.

Once the Newton-Raphson iterations have converged, we have obtained the updated gas internal energy and radiation energy, and we proceed to the next two steps of updating the coupling terms. We first update the radiation and gas momenta, accounting for the coupling term $\rho \kappa_F \mathbf{F}_r/c$. To do so compute the flux mean opacity $\kappa_F$ using the updated gas temperature. Then, following \cite{Skinner_2019}, we use a backward-Euler discretization of the radiation flux source term (modified to include a reduced speed of light):
\begin{align}
\vc{F}_r^{(t+1)} = \frac{\vc{F}_r^{(t)}}{1 + \rho \kappa_F \hat c \Delta t} \, .
\end{align}
In order to ensure momentum conservation when $\hat c = c$, we apply the difference in radiation flux in an equal and opposite manner to the gas momenta (as advocated by \citealt{Skinner_2019}):
\begin{align}
\Delta \vc{F}_r &\equiv \vc{F}_r^{(t+1)} - \vc{F}_r^{(t)} \, , \\
{(\rho \vc{v})}^{(t+1)} &= {(\rho \vc{v})}^{(t)} - \frac{\Delta \vc{F}_r}{\hat c c} \, .
\end{align}

The final step is to compute the work done by the radiation force on the gas. Since we are evolving the conserved variables, this term cannot be computed explicitly as written in \autoref{eq:rad_energy} without causing a significant error in the gas internal energy when the radiation force is stiff. We instead compute this term as the difference in gas kinetic energy over the timestep $\Delta E_{\text{kin}}$, then add this quantity to the total gas energy and subtract this quantity from the radiation energy:
\begin{align}
E^{(t+1)} &\leftarrow E_g^{(t+1)} + \left( E_{\text{kin}}^{(t)} + \Delta E_{\text{kin}} \right) \, , \\
E_r^{(t+1)} &\leftarrow E_r^{(t+1)} - \left( \frac{\hat c}{c} \right) \Delta E_{\text{kin}} \, ,
\end{align}
where $E^{(t+1)}$ denotes the total gas energy at the end of the timestep. This completes the update for all radiation-matter coupling terms.


\subsubsection{Level synchronization procedure}
\label{sssec:sync}

As explained by \cite{Berger:1989}, in an AMR calculation it is necessary to adjust the solution on the coarse AMR level following the solution on any refined level in order to maintain conservation of the evolved quantities (e.g., mass, momentum, energy). For hyperbolic equations evolved explicitly in time, this is traditionally done by saving the flux at the coarse-fine grid boundary in a `flux register' for both the flux computed on the fine level and the flux computed on the coarse level. In general, these fluxes are different due to the differing stencil used on the coarse and fine levels, and without correction, this would lead to a loss of conservation of energy (and any other conserved quantities). The flux register stores this mismatch, and in the synchronization step, adds the missing mass, momentum, or energy to the cells on the coarse level immediately adjacent to the coarse-fine boundary.

As noted by \cite{Howell_2003}, an implicit radiation update has additional difficulties in ensuring conservation, since radiation can propagate much further than a single grid cell on the coarse grid. Our radiation update is fully explicit, but we would like to advance each AMR level on the hydrodynamic timescale, rather than on the radiation timescale, so we have a similar long-range signal propagation difficulty. \cite{Rosdahl_2013} outline three possible solutions to the problem: i) perform the radiation solve after each coarse hydrodynamic step, keeping subcycling-in-time on refined levels (which would be very inaccurate), ii) use a single global timestep for all AMR levels, which allows one to advance the radiation solution on all levels in each radiation substep (which would be very computationally expensive, since in our applications of interest, the global timestep is typically limited by the timestep of the highest-resolution level) or iii) restrict the timestep for each level to the minimum of the radiation and hydrodynamic timesteps. In our code, we set the coarse timestep such that the number of radiation substeps per level is limited to a maximum value $N_{\text{sub,max}}$ in order to minimize the signal propagation distance from the coarse-fine boundaries. The flux mismatch at the coarse-fine boundaries is added to the immediately adjacent cells on the coarse grid at the end of each level advance. When $N_{\text{sub,max}} = 1$, our solution is identical to the flux synchronization method used in the \textsc{Ramses} AMR code \citep{Rosdahl_2013}. However, as a default we set the parameter $N_{\text{sub,max}}$ to $10$, which appears to be sufficient to avoid significant discontinuities in the radiation energy and flux at coarse-fine boundaries, but still allows for significant subcycling and thus a substantially lower computational cost. We use this value for all test problems shown in this work, but users are able to set this parameter as desired for either greater efficiency or greater consistency at refinement boundaries. When this parameter is too large, however, it is possible for the coarse level to fail to maintain positivity of the radiation energy or causality of the radiation flux.

\section{Test problems}
\label{section:tests}

We now proceed to describe a series of tests that we have conducted to verify \quokka's accuracy and convergence characteristics, starting with tests of the hydrodynamic subsystem (\autoref{ssec:hydro_tests}), followed by tests of the radiation transport and radiation-matter exchange subsystems (\autoref{ssec:radiation_tests}), and concluding with tests of coupled radiation-hydrodynamics (\autoref{ssec:radhydro_tests}).

Additional example problems and an automated test suite of $19$ test problems with checks against exact solutions are included with \textsc{Quokka}'s source code. We run this test suite for each commit and pull request in our GitHub repository. While continuous integration tests such as ours cannot guarantee bug-free software, this practice has flagged and prevented the introduction of several bugs during the development of \textsc{Quokka}. In order to maintain high software quality, we also run the commercial static code analyzer \textsc{SonarQube}\footnote{Available from SonarSource S.A, Switzerland via \url{https://www.sonarqube.org}. We have detected several bugs affecting solution correctness in other hydrodynamics codes using this tool.} on every commit and pull request.

\subsection{Hydrodynamics}
\label{ssec:hydro_tests}

For all our hydrodynamics tests, we disable the radiation portion of the code. These tests evaluate the hydrodynamic transport solver in isolation.

\subsubsection{Sound wave}
We compute the propagation of a sound wave in one dimension in order to measure the convergence of our numerical method to the exact solution as a function of spatial resolution, following the test described by \cite{Stone_2008}. With $\vc{U}$ denoting the vector of conserved variables $(\rho, \rho v_x, \rho v_y, \rho v_z, E)$, we initialize the simulations with the initial state $\vc{U}_0 + \delta \vc{U}$, where
\begin{align}
\delta \vc{U} = A \vc{R} \sin(2\pi x)
\end{align}
where $\vc{R} = (1,-1,1,1,1.5)$ is the right eigenvector of the linearized hydrodynamic system, and $\vc{U}_0$ is the background state with density $\rho = 1$, velocity $\vc{v} = 0$, and pressure $P = 1/\gamma$. We set the adiabatic index $\gamma = 5/3$ and the wave amplitude $A = 10^{-6}$. We simulate a periodic domain $x= 0$ to 1 and evolve the system for one wave period, allowing us to compute the error of the solution by direct comparison of the initial conditions and the final state of the simulation. We define the error vector
\begin{align}
(\Delta \vc{U})_{k} = \frac{1}{N_x} \sum_{i} \left| U_{i,k} - U_{i,k}^{0} \right|
\label{eq:convergence_error}
\end{align}
where $k$ denotes a component of each state vector, $\vc{U_i}$ is the vector of conserved variables in cell $i$ at the final timestep, and $\vc{U_i}^{0}$ is the vector of conserved variables in cell $i$ in the initial conditions. Each component $|\Delta U|_k$ is therefore the $L_1$ norm of the error of a component of the solution state.  We assess the accuracy of the solution based on the root-mean-square (rms) of the components of this error vector, denoted $||\Delta \vc{U}||$.

We run simulations using PPM reconstruction and a CFL number of $0.1$, using grid sizes from $N_x = 16$ to $N_x = 1024$. We show the error norm as a function of resolution in \autoref{fig:sound_wave}. For $N_x = 16$, we obtain $||\Delta \vc{U}|| = 1.0 \times 10^{-7}$, for $N_x = 128$, we obtain $||\Delta \vc{U}|| = 1.6 \times 10^{-9}$, and for $N_x = 1024$, we obtain $||\Delta \vc{U}|| = 1.7 \times 10^{-11}$. The results for our code are in excellent agreement with those from the \textsc{Athena} hydrodynamic solver (Figure 7 of \citealt{Stone_2008}). The $N_x^{-2}$ scaling of the error norm indicates that our hydrodynamic solver converges at second order in spatial resolution, as expected from the formal order of accuracy of the method.
\begin{figure}
    \includegraphics[width=\columnwidth]{wave_convergence.pdf}
    \caption{The error $||\Delta \vc{U}||_{2}$ in the solution (\autoref{eq:convergence_error}) for a sound wave as a function of spatial resolution per wavelength $N_x$. Black circles show numerical results, and the dashed line is a power law that scales as $N_x^{-2}$ normalised to the observed error at the smallest $N_x$.}
    \label{fig:sound_wave}
\end{figure}

\subsubsection{Contact discontinuity}
The HLLC Riemann solver has the property that it can resolve an isolated stationary contact discontinuity with infinite resolution \citep{Toro_2013}. The HLL solver, on the other hand, introduces a large amount of numerical diffusion for this problem (see Figure 10.20 of \citealt{Toro_2013}). To verify that our hydrodynamic implementation can maintain a perfect contact discontinuity, we simulate a system where the initial conditions have a left and right state separated with a discontinuity at $x = 0.5$. The left state is $\rho_L = 1.4$, $p_L = 1.0$, and the right state is $\rho_R = 1.0$ and $p_R = 1.0$. Since this is a pure contact discontinuity, the solution should not evolve from the initial state. We set the velocity to zero, and use an adiabatic index $\gamma = 1.4$. We evolve the solution numerically until $t = 2$. The error with respect to the correct solution is exactly zero.

\subsubsection{Stationary shock tube}
\label{section:shocktube}
Our next test is a stationary shock tube, which we set up using the parameters suggested on the website of F.X. Timmes\footnote{\url{http://cococubed.asu.edu/code_pages/exact_riemann.shtml}}. This shock tube problem is substantially more difficult to solve than the standard \cite{Sod_1978} shock tube test due to the larger jump in pressure and density at the discontinuity. We initialize left and right states with a discontinuity at $x_0 = 2$, with the left state $\rho_L = 10$, and $p_L = 100$, and the right state $\rho_R = 1$ and $p_R = 1$. The initial velocity is zero. We evolve the solution using a CFL number of 0.1 until $t = 0.4$ on a grid of 1000 cells on the domain $[0, 5]$. We use a small CFL number since the wave structure at the discontinuity creates waves that propagate faster than the linearized Roe eigenvalues would predict.

We show \quokka's results for this test in \autoref{fig:shocktube}. As for the sound wave test, we compute the $L_1$ error norm for each of the conserved variables, and then compute the root-mean-square of those error norms. The rms $L_1$ error norm divided by the rms norm of the exact solution is $1.12 \times 10^{-3}$. Inspecting the solution in \autoref{fig:shocktube}, we see that the agreement between the exact solution and the numerical solution is very good. The only noticeable differences are small oscillations near discontinuities in the derivative of the solution at $x \approx 2.4$ and near the density discontinuity at $x \approx 3.6$. We find that the shock flattening method of \cite{Miller_2002} is essential to produce a reasonable solution to this problem. Without it, we find unacceptably large post-shock oscillations (not shown).
\begin{figure}
    \includegraphics[width=\columnwidth]{hydro_shocktube_0.4000.pdf}
    \caption{A stationary shock tube problem (\autoref{section:shocktube}).}
    \label{fig:shocktube}
\end{figure}

\subsubsection{`LeBlanc' test}
We next carry out the `LeBlanc' shock tube test, originally published by \cite{Benson_1992} and further described by \cite{Pember_2001}. In this problem, we initialize a left state with $\rho_L = 1$ and $p_L = \frac{2}{3} \times 10^{-1}$, and a right state with $\rho_R = 10^{-3}$ and $p_R = \frac{2}{3} \times 10^{-10}$. We set the initial velocity to zero and use an adiabatic index $\gamma = 5/3$. This is an extreme shock tube that far exceeds any shock that may be encountered in any conceivable application, featuring a pressure jump of nine orders of magnitude, and is therefore an excellent test problem.  We evolve this simulation until $t = 6$ using a grid of 2000 cells and a CFL number of 0.1. The resulting state is shown in \autoref{fig:leblanc}. \cite{Pember_2001} highlight the difficulty of obtaining the correct specific internal energy in the solution for this test, but we find that \quokka~produces the correct shock location and specific internal energy, with the exception of a small overshoot at the shock location. The use of shock flattening is essential for this problem. Overall, the performance of our code on this problem is excellent.
\begin{figure}
    \includegraphics[width=\columnwidth]{hydro_leblanc_6.0000.pdf}
    \includegraphics[width=\columnwidth]{hydro_leblanc_eint_6.0000.pdf}
    \caption{The LeBlanc test problem.}
    \label{fig:leblanc}
\end{figure}

\subsubsection{Wave-shock interaction (Shu-Osher) test}
We show the Shu-Osher test in \autoref{fig:shu_osher}. Following the description of \cite{Shu_1989}, the initial conditions are, on the left side,
\begin{align}
\rho_L(x) &= 3.857143 \, , \\
v_L(x) &= 2.629369 \, , \\
P_L(x) &= 10.33333 \, ,
\end{align}
and the right-hand state is
\begin{align}
\rho_R(x) &= 1 + 0.2 \sin(5x) \, , \\
v_R(x) &= 0 \, , \\
P_R(x) &= 1 \, .
\end{align}
We compute a reference solution using \textsc{Athena++} \citep{Stone_2020} with the VL2 integrator, PPM reconstruction in the characteristic variables, and the HLLC Riemann solver on a grid of 1600 cells. Our solution is computed using PPM reconstruction (in the primitive variables), the RK2-SSP integrator, and the HLLC Riemann solver on a grid of 400 cells. The agreement is very good, with comparable resolution of the high-frequency features to the third-order essentially-non-oscillatory (ENO) scheme of \cite{Shu_1989}. When PLM reconstruction is used instead for the same number of grid cells, the high-frequency features are aliased (not shown; see also Figure 14 of \citealt{Shu_1989}), indicating a significantly higher effective resolution for PPM-based methods even in the presence of shocks.
\begin{figure}
    \includegraphics[width=\columnwidth]{hydro_shuosher.pdf}
    \caption{The Shu-Osher wave-shock interaction problem.}
    \label{fig:shu_osher}
\end{figure}

\subsubsection{Slow-moving shock}
We show a slow-moving shock in \autoref{fig:sms} using the parameters from \cite{Jin_1996}, where $\rho_L = 3.86$, $(\rho v)_L = -3.1266$, and $E_L = 27.0913$, and the right-side state $\rho_R = 1.0$, $(\rho v)_R = -3.44$, and $E_R = 8.4168$, with $\gamma = 1.4$. This corresponds to the shock jump moving to the right with a velocity $v_{\text{shock}} = 0.1096$. For a CFL number of $0.2$, this corresponds to the shock taking $\sim 250$ timesteps to move across a single cell. This may not be a common scenario for our applications, but it may occur in a protostellar accretion shock, for instance. The quality of the solution is again significantly improved by shock flattening. The post-shock oscillations for slow-moving shocks may still be present with first-order reconstruction \citep{Jin_1996,Lee_2011}, so it is difficult to completely eliminate. We also find that adding a small amount of artificial viscosity does not significantly reduce the oscillations. A modification to PPM reconstruction based on a characteristic wave decomposition succeeds in significantly reducing this oscillation \citep{Lee_2011}, which we may consider implementing in a future version of the code.
\begin{figure}
    \includegraphics[width=\columnwidth]{hydro_sms_1.0000.pdf}
    \caption{A slow-moving shock test problem.}
    \label{fig:sms}
\end{figure}

\subsubsection{Strong rarefaction}
We next test the performance of our code on the 1-2-3 problem of \cite{Einfeldt_1991}, which features a strong rarefaction and is designed to induce failures in approximate Riemann solvers. The initial conditions consist of left and right states with equal density and energy, $\rho_L = \rho_R = 1$ and $E_L = E_R = 3$, and equal magnitude but oppositely-directed velocities, $(\rho v)_L = -2$, $(\rho v)_R = 2$. We evolve the system to $t=0.15$, using a CFL number $0.8$ and a grid of $N_x = 100$ cells, and show the resulting state in \autoref{fig:vacuum}. We obtain the exact solution to which we compare the \quokka~result using an exact Riemann solver. We find that the solution for the density profile is very close to the exact solution, except for a small discrepancy at the lowest density near $x = 0.5$. However, the most difficult aspect of this problem is obtaining the correct specific internal energy. Our results compare favorably with other solutions obtained with approximate Riemann solvers, where factors of two or three errors are obtained near $x \approx 0.5$ \citep{Toro_2013}. Obtaining the correct specific internal energy in the lowest density part of the flow may require the adaptive use of an exact Riemann solver in near-vacuum regions. Nonetheless, our code is stable and well-behaved for this problem.
\begin{figure}
    \includegraphics[width=\columnwidth]{hydro_vacuum_0.1500.pdf}
    \includegraphics[width=\columnwidth]{hydro_vacuum_eint_0.1500.pdf}
    \caption{A strong rarefaction test problem.}
    \label{fig:vacuum}
\end{figure}

\subsubsection{Kelvin-Helmholtz problem}
We illustrate this problem in \autoref{fig:kh_zoom}\dots
\begin{figure}
    \includegraphics[width=0.9\columnwidth]{quokka_full.pdf}
    \includegraphics[width=0.9\columnwidth]{quokka_zoom.pdf}
    \includegraphics[width=0.9\columnwidth]{quokka_zoom2.pdf}
    \caption{A Kelvin-Helmholz problem with 4 levels of refinement.}
    \label{fig:kh_zoom}
\end{figure}

\subsubsection{Liska-Wendroff Implosion}
To be written\dots

\subsection{Radiation}
\label{ssec:radiation_tests}

For our radiation tests we disable the hydrodynamic part of the code and only use the radiation transport and gas-radiation exchange updates. These tests evaluate the accuracy of those portions of the code.

\subsubsection{Marshak wave}
We next compute a Marshak wave \citep{Marshak_1958}. The problem consists of a uniform gas a constant density $\rho = 10 \gm \cm^{-3}$ and constant opacities $\kappa_P = \kappa_R = 577 \cm^{2} \gm^{-1}$. The gas has a uniform initial temperature of $10^4 \Kelvin$, but at $t=0$ we impose on the left-hand side of the domain a boundary condition consisting of a half-isotropic flux with a radiation temperature of $3.481334 \times 10^6 \Kelvin$. The radiation drives a wave of heat into the gas. Following \cite{Su_1996}, we set the gas heat capacity at constant volume $C_v$ so a functional form that makes it possible to linearize the matter-radiation coupling terms, and thus obtain a semi-analytic solution:
\begin{align}
C_v &\equiv \frac{\partial E_{\text{int}}}{\partial T} = \alpha T^3 \, ,
\label{eq:heat_capacity}
\end{align}
where $E_{\text{int}} = (\alpha / 4) \, T^4$, $\alpha = 4 a_r / \epsilon$ and $\epsilon = 1$. With this heat capacity, \cite{Su_1996} obtain a semi-analytic quadrature solution of the radiation diffusion equation for this problem as a function of $\epsilon$. We evolve the solution until time $t = \tau / (\epsilon c \rho \kappa)$ where $\tau = 10$, using a simulation domain on the interval $[0 \cm, 3.466205 \times 10^{-3}\cm]$ resolved by grid of $N_x = 400$ cells. We do not use a reduced speed of light for this test.

Since we solve the moment equations, rather than just the diffusion equation, we do not expect our numerical solution to agree with the \citeauthor{Su_1996} solution at the leading edge of the wave, where our code respects causality and restricts the propagation speed of the wave to $c$; this constraint is violated in the diffusion approximation that \citeauthor{Su_1996} adopt. However, we can still compare to their solution in the region where $F_r \ll cE_r$ and diffusion is a good approximation. In this region, we obtain excellent agreement with the semi-analytic solution, as shown in \autoref{fig:marshak}. Note that the difference between our numerical solution and the ``exact'' solution at $x \gtrsim 3\times 10^{-3}$ cm is not an error in our solution. Rather, it is a result our code properly capturing the finite speed of light, while the semi-analytic solution does not.
\begin{figure}
    \includegraphics[width=\columnwidth]{marshak_wave_cgs_gastemperature.pdf}
    \caption{A Marshak wave test problem \citep{Su_1996}.}
    \label{fig:marshak}
\end{figure}

\subsubsection{Su-Olson problem}
We next compute a problem involving radiation penetrating a cold medium but with an internal radiation source rather than a radiation source at the boundary. This problem is defined in dimensionless units where $a_r = c = 1$, with opacities $\kappa_P = \kappa_R = 1$, a constant density $\rho = 1$, and a radiation source
\begin{align}
S(x,t) =
\begin{cases}
    Q \, a_r T_H^4 & 0 \leq x < x_0 \, \text{and} \, t < t_0 \, , \\
    0 & x \geq x_0 \, \text{or} \, t \geq t_0 \, ,
\end{cases}
\end{align}
where we have a normalisation factor $Q = (2 x_0)^{-1}$, radiation source temperature $T_H = 1$, and spatial extent of the source $x_0 = 0.5$ and temporal extent $t_0 = 10$. The initial radiation and gas energies are zero in the idealized problem, but we set them to $10^{-10}$ in our simulation since the radiation solver requires nonzero gas and radiation energies. The gas velocity is zero. We adopt reflecting boundary conditions on the domain $[0, 30]$ on a grid of $N_x = 1500$ cells. We do not reduce the speed of light for this test.

When using the heat capacity given by \autoref{eq:heat_capacity}, a semi-analytic solution of the angle-dependent transport equation may be obtained with a Fourier-Laplace transform \citep{Su_1997}. This solution assumes that $\vc{v} = 0$ at all times, so we drop all $v/c$ terms for this problem. We show our numerical solution using CFL number $0.4$ at time $t = 10$  in \autoref{fig:suolson}; for comparison, we also show the exact transport solution and the exact diffusion solution. We find that with the \cite{Levermore_1984} closure, we obtain a solution in between the diffusion solution and the transport solution. While it makes little difference at $t = 10$, we find better agreement with the transport solution at earlier times when using the \cite{Minerbo_1978} closure (not shown). In this problem, some regions near the internal radiation source (located at $0 \leq x < 0.5$) have Eddington factors $\chi < 1/3$, which cannot be represented by any local closure of the form given by \autoref{eq:M1_closure}.  Nonetheless, we obtain a solution that is more accurate than one would obtain by using a radiation diffusion equation.
\begin{figure}
    \includegraphics[width=\columnwidth]{SuOlsonTest.pdf}
    \caption{The Su-Olson test problem \citep{Su_1997}.}
    \label{fig:suolson}
\end{figure}

\subsubsection{Radiation-matter energy exchange}
\label{section:equilibrium}
We next isolate the implicit matter-radiation energy exchange solver by solving a problem with no transport. Following \citet{Turner_2001}, we set up a uniform domain with periodic boundary conditions, where the gas and radiation are initially out of thermal equilibrium. The initial radiation energy density is $E_r = 10^{12} \, \text{erg} \, \text{cm}^{-3}$ and, the initial gas energy density $E_g = 10^2 \, \text{erg} \, \text{cm}^{-3}$. The density $\rho = 10^{-7} \, \text{g} \, \text{cm}^{-3}$ and the specific opacity $\kappa_P = 1.0 \, \cm^{2} \gm^{-1}$. Rather than using a constant heat capacity (as \citealt{Turner_2001} do) we use the heat capacity given by \autoref{eq:heat_capacity}, which allows us to obtain an algebraic solution for the matter temperature $T$ as a function of time $t$:
\begin{align}
T^4 &= \left( T_{0}^4 - \frac{\hat c}{c} \tilde E_0 \right) \, \exp \left[ -\frac{4}{\alpha} \left( a_r + \frac{\hat c}{c} \frac{\alpha}{4} \right) \kappa \rho c t \right] \, + \, \frac{\hat c}{c} \tilde E_0 \, .
\end{align}
where $E_0 = E_{g} + (c/\hat c) \, E_{r}$ and $\tilde E_0 = E_0 \left[ a_r + (\hat c / c) (\alpha / 4) \right]^{-1}$ are constant as a function of time. Taking the limit $t \rightarrow \infty$, we immediately see that the equilibrium temperature $T_{\text{eq}}$ is modified whenever $\hat c \neq c$, contrary to previous claims in the literature:
\begin{align}
T_{\text{eq}}^4 &= \frac{\hat c}{c} \tilde E_0 = \frac{\hat c}{c} E_0 \left[ a_r + \left(\frac{\hat c}{c} \right) \frac{\alpha}{4} \right]^{-1} \, .
\end{align}
Fundamentally, this occurs whenever RSLA is employed (and $\hat c \neq c$) because the quantity $E_0 = E_g + (c / \hat c) \, E_r$ is conserved in this problem, \emph{not} the total energy $E_{\text{tot}} = E_g + E_r$. This is a generic failing of the RSLA, which does not conserve total energy. However, in practice when the boundary conditions are such that the quantity $E_0$ is \emph{not} conserved, the physically correct steady-state solution may still be obtained -- this is the situation for all the radiation-hydrodynamics test problems we present in \autoref{ssec:radhydro_tests}, and is also the situation for most applications of interest.

To test \quokka's ability to recover the analytic solution, we simulate the problem using a constant timestep $\Delta t = 10^{-8} \s$ until $t = 10^{-2} \s$. We use a reduced speed of light $\hat c = 0.1 c$. We show the time evolution of the matter and radiation temperatures in \autoref{fig:radcoupling}. We find that the numerical solution agrees with the exact solution to better than one part in $10^{5}$ at each timestep.
\begin{figure}
    \includegraphics[width=\columnwidth]{radcoupling_rsla.pdf}
    \caption{A radiation-matter coupling test (\autoref{section:equilibrium}).}
    \label{fig:radcoupling}
\end{figure}

\subsubsection{Shadow test}
To be written\dots

\subsubsection{Beam test}
To be written\dots

\subsection{Radiation hydrodynamics}
\label{ssec:radhydro_tests}

Our final suite of tests use the full suite of physics in \quokka, and involve coupled radiation and hydrodynamics.

\subsubsection{Radiation pressure tube}
Our first radiation-hydrodynamic test is the the radiation pressure tube problem of \cite{Krumholz_2007}. This problem is designed to show that the radiation pressure gradient can stably balance the gas pressure gradient both in the regime where radiation pressure dominates and the in the regime where gas pressure dominates for a problem where the optical depth is sufficiently large that the radiation is in the equilibrium diffusion regime. We adopt the opacities $\kappa_P = \kappa_R = 100 \cm^2 \gm^{-1}$, mean molecular weight $\mu = 2.33 \, m_H$, and adiabatic index $\gamma = 5/3$. The exact steady-state solution in the diffusion approximation is given by the solution to the differential equations
\begin{align}
\frac{d \rho}{dx} &= -\frac{\mu}{k_B T} \left( \frac{k_B}{\mu} \rho + \frac{4}{3} a_r T^3 \right) \frac{dT}{dx} \, , \\
\frac{d^2 T}{dx^2} &= -\frac{3}{T} \left(\frac{dT}{dx}\right)^2 + \frac{1}{\rho} \frac{d\rho}{dx} \frac{dT}{dx} \, ,
\end{align}
where the left-side temperature, density, and density gradient are $T_0 = 2.75 \times 10^7 \K$, $\rho_0 = 1.0 \gm \cm^{-3}$, and ${d\rho_0}/{dx} = 0.005 \gm \cm^{-4}$. We solve this equation on the domain $[0 \cm, 128 \cm]$ in order to obtain the initial conditions for this problem. The left and right side initial conditions are adopted as Dirichlet boundary conditions for our simulation. The reduced speed of light $\hat c$ is set to $10 \, c_{s,0} \approx 4.03 \times 10^8 \cm \s^{-1}$, where $c_{s,0}$ is the sound speed at the left boundary.

After evolving for a sound crossing time $t = L_x / c_{s,0} \approx 3.177 \times 10^{-6} \s$ with a CFL number of $0.4$ on a grid of 128 cells, we obtain the numerical solution shown in \autoref{fig:radiation_pressure_tube}. Our numerical solution agrees with the initial conditions (obtained from the exact diffusion solution) to better than $0.2$ per cent. Since the boundary conditions do not require conservation of the quantity $E_0$ (see \autoref{section:equilibrium}), we find that we are able to obtain the physically correct solution even when $\hat c \neq c$.

\begin{figure}
    \includegraphics[width=\columnwidth]{radiation_pressure_tube.pdf}
    \caption{The temperature for the radiation pressure tube \citep{Krumholz_2007}. The radiation and gas temperatures should be identical. The temperature for the exact diffusion solution is shown in the black circles. The temperatures agree to within 0.2 per cent.}
    \label{fig:radiation_pressure_tube}
\end{figure}

\subsubsection{Optically-thin radiation-driven wind}
In order to test the radiation-gas momentum coupling in the optically-thin limit, we next simulate a problem based on the radiation-inhibited Bondi accretion test of \cite{Krumholz_2007}. Since we have not yet implemented sink particles or gravity, we use the plane-parallel modification of \cite{Skinner_2013}. The opacities are $\kappa_P = 0$ and $\kappa_R = 5 \cm^2 \gm^{-1}$. The gas is isothermal with a sound speed of $c_T = 0.2 \km \s^{-1}$. The optical depth across the domain is $10^{-6}$. The initial Mach number $\mathcal{M}_0$ is $1.1$. The radiation flux $F_{r,0}$, acceleration $g_0$, and scale height $L$ are
\begin{align}
F_{r,0} = \rho_0 c_T c / \tau \, , \\
g_0 = \kappa_R F_{r,0} / c \, , \\
L = c_T^2 / g_0 \, .
\end{align}
where $\rho_0$ is the density at the base of the wind. We choose $\rho_0 = 3.897212 \times 10^{-19} \gm \cm^{-3}$. The length of the computational domain is one scale height $L$. As \cite{Skinner_2013} show, the steady-state solution for the Mach number $M$ is implicitly given by the Bernoulli equation
\begin{align}
\frac{1}{2} \mathcal{M}_0^2 = \frac{1}{2} \mathcal{M}^2 + \log \left({\frac{\mathcal{M}_0}{\mathcal{M}}}\right) - \left( \frac{x}{L} \right) \, ,
\end{align}
where $x$ is the position above the base of the wind. We neglect gravitational forces in this problem. The solution of the equation above is used for the initial conditions, and Dirichlet boundary conditions are adopted using the values of the solution at the boundaries. An exact isothermal Riemann solver is used for this simulation. Using a CFL number of 0.4, we evolve until $t = 10 \, (L/c_T)$.
In \autoref{fig:wind}, we show the exact solution and our numerical solution for the Mach number as a function of height, finding excellent agreement.
\begin{figure}
    \includegraphics[width=\columnwidth]{radiation_force_tube_vel.pdf}
    \caption{A radiation-driven wind.}
    \label{fig:wind}
\end{figure}

\subsubsection{Subcritical radiative shock}
\label{section:radshock}
We next simulate a subcritical radiative shock, following the set-up used by \cite{Skinner_2019} with the dimensionless parameters for the Mach $\mathcal{M} = 3$ example given by \cite{Lowrie_2008}. We scale to cgs units with the opacities $\kappa_P = \kappa_R = 577 \cm^{2} \gm^{-1} (1 \gm \cm^{-3} / \rho)$, mean molecular weight $\mu = m_{\rm H}$, and adiabatic index $\gamma = 5/3$. The left-side state consists of $\rho_L = 5.69 \gm \cm^{-3}$, velocity $v_L = 5.19 \times 10^7 \cm \s^{-1}$, and temperature (gas and radiation) $T_L = 2.18 \times 10^6 \Kelvin$. The right-side state is $\rho_R = 17.1 \gm \cm^{-3}$, $v_R = 1.73 \times 10^7 \cm \s^{-1}$, and $T_R = 7.98 \times 10^6 \K$. These states are also used as Dirichlet boundary conditions for the simulation. In order to exactly match the assumptions used in the semi-analytic solution of \cite{Lowrie_2008}, we use the Eddington approximation (i.e., $\textsf{P}_r = (1/3) E_r \textsf{I}$) to close the radiation pressure tensor for this problem.\footnote{We provide a \textsc{Python} code that computes the semi-analytic solution for radiative shocks using the Eddington approximation \citep{Lowrie_2008} in our \faGithub\href{https://github.com/BenWibking/quokka-code}{GitHub repository}.} Following \cite{Skinner_2019}, we use a reduced speed of light $\hat c = 10(v_L + c_{s,L})$, where $c_{s,L}$ is the adiabatic sound speed of the left-side state.  We use a CFL number of 0.4 and evolve until $t = 10^{-9} \s$ on a grid of $512$ cells on the domain $[0 \cm, 0.01575 \cm]$, with the discontinuity placed between the left- and right-side states $x_0 = 0.0130 \cm$. The shock drifts $1.5$ per cent of the domain length to the right from the location of the initial discontinuity, which may be due to a combination of the initial numerical transient and our use of the asymptotic states as boundary conditions, rather than the exact states expected at a finite distance from the shock location. This makes the steady-state location of the shock on the simulation grid not well-defined. After accounting for this drift, the agreement between the numerical and semi-analytic solution is excellent, as shown in \autoref{fig:radshock}. We find the that the relative error of the gas temperature in $L_1$ norm is $0.4$ per cent, which is at least as good as the solution of \cite{Skinner_2019} for the same spatial resolution.  In this problem, we find that using shock flattening is essential to obtain a non-oscillatory temperature structure for the Zel'dovich spike (the gas temperature discontinuity shown in \autoref{fig:radcoupling}; \citealt{Zeldovich_1967}).
\begin{figure}
    \includegraphics[width=\columnwidth]{radshock_cgs_temperature.pdf}
    \caption{A subcritical radiative shock with $\mathcal{M} = 3$.}
    \label{fig:radshock}
\end{figure}

\subsubsection{Radiation-driven dust shell}
\label{section:shell}
To be written\dots

\section{Performance and scaling}
\label{section:performance}
\subsection{Weak scaling}
Weak scaling results are shown in \autoref{table:weak_hydro_scaling} and \autoref{table:weak_radhydro_scaling}.

\begin{table}
\begin{tabular}{l|r|r|r|r|r|r}\hline
Nodes & GPUs & Mzones/GPU & Scaling efficiency\\\hline
\csvreader
    {weak_scaling_hydro.csv}{1=\nodes,2=\gpus,3=\mzones,4=\mzonespergpu,5=\gpufill,6=\scaling,7=\scalingnode}
    {\nodes & \gpus & \mzonespergpu & \scalingnode \\}
\end{tabular}
\caption{Weak scaling efficiency for hydrodynamics as a function of the number of GPUs for a Sedov blast wave with periodic boundary conditions.}
\label{table:weak_hydro_scaling}
\end{table}

\begin{table}
\begin{tabular}{l|r|r|r|r|r|r}\hline
Nodes & GPUs & Mzones/GPU & Scaling efficiency\\\hline
\csvreader
    {weak_scaling_radhydro.csv}{1=\nodes,2=\gpus,3=\mzones,4=\mzonespergpu,5=\gpufill,6=\scaling,7=\scalingnode}
    {\nodes & \gpus & \mzonespergpu & \scalingnode \\}
\end{tabular}
\caption{Weak scaling efficiency for radiation hydrodynamics as a function of the number of GPUs for the radiation-driven shell test (\autoref{section:shell}) with periodic boundary conditions.}
\label{table:weak_radhydro_scaling}
\end{table}

\subsection{Strong scaling with AMR}
Shown in \autoref{table:strong_scaling}.
\begin{table}
\begin{tabular}{l|r|r|r|r|r|r}\hline
Nodes & GPUs & Mzones/GPU & $\left(\frac{\text{Cells}}{\text{GPU}}\right)^{1/3}$ & Scaling efficiency\\\hline
\csvreader
    {strong_scaling.csv}{1=\nodes,2=\gpus,3=\mzones,4=\mzonespergpu,5=\gpufill,6=\scaling,7=\cellspergpu}
    {\nodes & \gpus & \mzonespergpu & \cellspergpu & \scaling \\}
\end{tabular}
\caption{Strong scaling efficiency for radiation hydrodynamics as a function of the number of GPUs for the radiation-driven shell test (\autoref{section:shell}) with periodic boundary conditions on a base grid of $256^3$ cells and 2 levels of refinement.}
\label{table:strong_scaling}
\end{table}

\section{Discussion and Conclusions}
\label{section:discussion}
\subsection{Range of applicability}
\subsection{Future extensions}

The future is bright for radiation hydrodynamics on GPUs\dots

\section*{Acknowledgements}

This research was supported by the Australian Research Council through its Discovery Projects and Future Fellowship Funding Schemes, awards DP190101258 and FT180100375. This research was undertaken with the assistance of resources and services from the National Computational Infrastructure (NCI), which is supported by the Australian Government.

\emph{Software:} AMReX \citep{the_amrex_development_team_2021_5363443},
matplotlib \citep{Hunter:2007},
numpy \citep{harris2020array}, yt \citep{Turk11a}.

%%%%%%%%%%%%%%%%%%%%%%%%%%%%%%%%%%%%%%%%%%%%%%%%%%
\section*{Data Availability}
The source code and entire commit history for \textsc{Quokka} is hosted in this public \faGithub\href{https://github.com/BenWibking/quokka-code}{GitHub repository}. The version of the source code used to produce the results in this paper as well as the output files at the final timestep for the simulations shown in the Figures are permanently archived at Zenodo DOI:\textbf{XXX}.

%%%%%%%%%%%%%%%%%%%% REFERENCES %%%%%%%%%%%%%%%%%%

\bibliographystyle{mnras}
\bibliography{quokka} % if your bibtex file is called example.bib

%%%%%%%%%%%%%%%%%%%%%%%%%%%%%%%%%%%%%%%%%%%%%%%%%%

%%%%%%%%%%%%%%%%% APPENDICES %%%%%%%%%%%%%%%%%%%%%

\appendix
\section{Asymptotic diffusion correction}
\label{appendix:asymptotic_correction}

%%%%%%%%%%%%%%%%%%%%%%%%%%%%%%%%%%%%%%%%%%%%%%%%%%


% Don't change these lines
\bsp	% typesetting comment
\label{lastpage}
\end{document}

% End of mnras_template.tex
